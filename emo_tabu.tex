\documentclass[a4paper,10pt]{llncs}
\usepackage[utf8]{inputenc}

%opening
\title{Using a parallel tabu search to approximate uniform design}
\author{Alberto Rodr\'iguez S\'anchez\inst{1}\and
Antonin Ponsich\inst{1} \and
Antonio L\'opez Jaimes\inst{2} \and
Sa\'ul Zapotecas Mart\'inez\inst{2}
}

\authorrunning{Rodr\'iguez S.  et al.}

\institute{Dpto. de Sistemas, Universidad Aut\'onoma Metropolitana Azcapotzalco
\email{\{ars,aspo\}@correo.azc.uam.mx}\\
\and
Dpto. de Matem\'aticas Aplicadas y Sistemas, Universidad Aut\'onoma Metropolitana Cuajimalpa\\
\email{alopez@correo.cua.uam.mx, saul.zapotecas@gmail.com}}

\begin{document}

\maketitle

\begin{abstract}
In Multi-objective Optimization (MO), diversity assessment is one of the most important concern in order to produce an approximated set of solutions evenly distributed over 
the Pareto front. To deal with this issue, recent algorithms such as MOEA/D\cite{4358754} make use of a uniformly scattered set of reference points/vectors that indicates 
search directions in the objective space. 
This issue becomes critical in Many-objective Optimization, promoting the development of many generation techniques having a rich underlying theory mainly arising from chemistry and statistics areas.

Among these methods, the Uniform Design (UD) is based on the minimization of a discrepancy metric, which measures how well equidistributed the points are in a sample space. %, when the set of points is perfectly equidistributed in the sample space a discrepancy metric is equal to 0. %   QUÉ REPRESENTA LA MÉTRICA DE DISCREPANCIA.
Of particular interest in this work is the centered $L_2$ discrepancy metric proposed in \cite{fang2002centered}.
An exponentially increasing number of candidate sets can be generated using the Good Lattice Point (GLP) technique, which involves a huge computational 
cost (memory and time). It was demonstrated that the problem of finding a uniform design under a given discrepancy metric is NP-hard when the number of runs, $n \rightarrow \infty$ and the number of factors, $s > 1$.

In order to solve this optimization problem, a parallel Tabu Search (TS) is implemented in this work. A specific feature is the tabu list that only reports
``generator parameters'', which are the input needed by the GLP algorithm to generate the set of final 
reference points.
The best reference sets founded by TS were subsequently used to solve two classical MO problems with MOEA/D (using the Tchebycheff scalarizing function).
The results are compared with those of the Simplex Lattice Design (SLD) in terms of the Hypervolume (HV) and $\Delta$-diversity indicators.
Results highlight that the UD allows a significant improvement, particularly regarding diversity and when the number of objectives increases.
\end{abstract}

\section{Introduction}

\section{Overview on weight vector design for reference-based MOEAs}
\subsection{Based on mixture designs}
\subsection{Based on discrepancy functions}

\section{DU : problem statement }
\subsection{Good lattice point}
\subsection{Discrepancy functions}
\subsection{$CD_2$ function and optimization problem for UD}

\section{Solution technique : parallel Tabu Search}
\subsection{Decision variables and codification}
\subsection{Neighborhood and Tabu list}
\subsection{Specific features}


\section{Computational experiments and discussion}
\subsection{Uniform designs obtained}
\subsection{Testing on classical MO instances}

\section{Conclusions}


\bibliographystyle{splncs04}
\bibliography{bibliografia}

\end{document}
