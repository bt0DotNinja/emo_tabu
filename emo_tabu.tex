\documentclass[a4paper,10pt]{llncs}
\usepackage[utf8]{inputenc}
%\usepackage{amsmath}
\usepackage{graphicx}
\usepackage[ruled,vlined,lined,linesnumbered,algosection,spanish]{algorithm2e}
%\theoremstyle{definition}


%opening
\title{Using a parallel tabu search to approximate uniform design}
\author{Alberto Rodr\'iguez S\'anchez\inst{1}\and
Antonin Ponsich\inst{1} \and
Antonio L\'opez Jaimes\inst{2} \and
Sa\'ul Zapotecas Mart\'inez\inst{2}
}

\authorrunning{Rodr\'iguez S.  et al.}

\institute{Dpto. de Sistemas, Universidad Aut\'onoma Metropolitana Azcapotzalco
\email{\{ars,aspo\}@correo.azc.uam.mx}\\
\and
Dpto. de Matem\'aticas Aplicadas y Sistemas, Universidad Aut\'onoma Metropolitana Cuajimalpa\\
\email{alopez@correo.cua.uam.mx, saul.zapotecas@gmail.com}}

\begin{document}

\maketitle

\begin{abstract}
In Multi-objective Optimization (MO), diversity assessment is one of the most important concern in order to produce an approximated set of solutions evenly distributed over 
the Pareto front. To deal with this issue, recent algorithms such as MOEA/D\cite{4358754} make use of a uniformly scattered set of reference points/vectors that indicates 
search directions in the objective space. 
This issue becomes critical in Many-objective Optimization, promoting the development of many generation techniques having a rich underlying theory mainly arising from chemistry and statistics areas.

Among these methods, the Uniform Design (UD) is based on the minimization of a discrepancy metric, which measures how well equidistributed the points are in a sample space. %, when the set of points is perfectly equidistributed in the sample space a discrepancy metric is equal to 0. %   QUÉ REPRESENTA LA MÉTRICA DE DISCREPANCIA.
Of particular interest in this work is the centered $L_2$ discrepancy metric proposed in \cite{fang2002centered}.
An exponentially increasing number of candidate sets can be generated using the Good Lattice Point (GLP) technique, which involves a huge computational 
cost (memory and time). It was demonstrated that the problem of finding a uniform design under a given discrepancy metric is NP-hard when the number of runs, $n \rightarrow \infty$ and the number of factors, $s > 1$.

In order to solve this optimization problem, a parallel Tabu Search (TS) is implemented in this work. A specific feature is the tabu list that only reports
``generator parameters'', which are the input needed by the GLP algorithm to generate the set of final 
reference points.
The best reference sets founded by TS were subsequently used to solve two classical MO problems with MOEA/D (using the Tchebycheff scalarizing function).
The results are compared with those of the Simplex Lattice Design (SLD) in terms of the Hypervolume (HV) and $\Delta$-diversity indicators.
Results highlight that the UD allows a significant improvement, particularly regarding diversity and when the number of objectives increases.
\end{abstract}

\section{Introduction}

\section{Overview on weight vector design for reference-based MOEAs}

\subsection{Mixture design}

Experiments with mixtures are experiments in which the variants are proportions of ingredients in a mixture. An example is an experiment for determining the proportion of ingredients in a polymer mixture that will produce plastics products with the
highest tensile strength. Similar experiments are very commonly encountered in industries. Designs for deciding how to mix the ingredients are called experimental designs with mixtures.

A design of $N$ runs for mixtures of $m$ ingredients is a set of $N$ points in the domain:

\begin{equation}
T_m = \{ ( \lambda_1, \dots, \lambda_m ):\lambda_j \geq 0, j=1, \dots,m, \lambda_1 + \dots + lambda_m  = 1 \}
\end{equation}

The original MOEA/D\cite{4358754} just adopts the simplex-lattice design to set the aggregation coefficient vectors. There are at least two problems can be found with that design:
\begin{itemize}
 \item The experimental points are not very uniformly distributed on the experimental domain $T_m$.
 \item There are too many experimental points at the boundary of the experimental domain, i.e, the unitary simplex.
\end{itemize}

\subsection{Based on mixture designs}

a lot of work which appeared in the statistical literature proposed many kinds of designs. Scheff\'e introduced the simplex-lattice designs and the corresponding
polynomial models (cita de Scheff\'e simplex lattice,). Later he introduced an alternative design, the simplex-centroid design ( citar Experiments with mixtures), to the general simplex-lattice. Cornell gave a suggestion of axial design and
he gave a comprehensive review of nearly all the statistical articles on designs of experiments with mixtures and data analysis (buscar cita en Experiments with mixtures, designs, models, and the analysis of
mixture data).

\subsubsection{Simplex Lattice}

% TODO metodo de generacion y propieades

% Referencia: Experiments with mixtures, Scheffe 
% Referencia: Normal-Boundary Intersection: A New Method for Generating the Pareto Surface in Nonlinear Multicriteria (Das&Dennis)

\subsubsection{Two layer simplex Lattice}

% TODO (cita de Deb) 

\subsubsection{Simplex-centroid design}

% TODO metodo de generacion y propieades

\subsubsection{Axial design}

% TODO metodo de generacion y propieades

\subsection{Based on discrepancy functions} % Preguntar si es requerido, 

% referencias del Paper de Saul

% TODO Teorema de proyeccion en el simplejo


\section{UD : problem statement } % Discutir



\subsection{Good lattice point}
\subsection{Discrepancy functions}
\subsection{$CD_2$ function and optimization problem for UD}

% TODO Definicion en  FANG. Uniform designs and their application in industry

% TODO SPLIT equation

\begin{equation}
\mathop{\mathrm{argmin}} CD_2 = \left(\frac{13}{12}\right)^m - \left(\frac{2}{N}\right) \sum_{k=1}^N \prod_{i=1}^m \left(1+\frac{1}{2} |X_{ki}-\frac{1}{2}| - \frac{1}{2} |X_{ki}-\frac{1}{2}|^2  \right) + \left(\frac{1}{N^2}\right) \sum_{k=1}^N \sum_{j=1}^N \prod_{i=1}^m \left(1+\frac{1}{2} |X_{ki}-\frac{1}{2}| + \frac{1}{2} |X_{ji}-\frac{1}{2}| - \frac{1}{2} |X_{ki}-X_{ji}| \right) 
\end{equation}


 where:
 \begin{itemize}
  \item $X$ is the induced Matrix of a U-type design.
  \item $N$ Rows of $X$.
  \item $m$ Columns of $X$.
 \end{itemize}

\section{Solution technique : parallel Tabu Search}

\begin{algorithm}[H]
     \caption{Parallel Tabu Search}
    %\SetLine
     \KwData{$IterMax$: Stop Condition, $W$: workers, $T_{max}$: Tabu List size, $N$: Number of points wanted, $dim$: number of dimensions}
     \KwResult{$NUD$:Nearly Uniform Design}
     
 bag $\leftarrow$ SearchCoprime($N$)\;
 Solutions $\leftarrow$ RandomDisjoint(bag,$dim$,$W$)\;
 bestdis  $\leftarrow \infty$, bestsol  $\leftarrow \emptyset$, TabuList $\leftarrow \emptyset$ \;
 
 \While {$\neg$Stopping($IterMax$)}{
	\tcc{Parallel For}
	\For {$i:=1$ a $W$}{
	$U_i$ $\leftarrow$ GLP($Solutions_i$) \;
	$X_i$ $\leftarrow$ InducedMatrix($U_i$) \;
	$discrepancy_i$ $\leftarrow$ $CD_2(X_i)$ \;
        \If {$discrepancy_i < bestdis_i$ and $Solutions_i$ $\not\in$ TabuList}{
		      $bestsol_i$ $\leftarrow$ $Solutions_i$\;
		      $bestdis_i$ $\leftarrow$ $discrepancy_i$\;
		      TabuList $\leftarrow$ TabuList $\cup$ $Solutions_i$\;
		      }
	\tcc{Critical Zone}
	$Solutions_i$ $\leftarrow$ UpdateSol($Solutions_i$,bag)\;
 	\If {$|TabuList|$ $>$ $T_{max}$}{
 	  TabuList $\leftarrow$ MaintainTabuList(TabuList) \;
 	}
 }
 
 $NUD$ $\leftarrow$ $\min(Solutions, bestdis)$ \; 
 }
 \Return($NUD$)
\end{algorithm}

\subsection{Decision variables and codification}

% TODO Beto

\subsection{Neighborhood and Tabu list}

% TODO Beto

\subsection{Specific features}

% TODO Beto

\section{Computational experiments and discussion}

\subsection{Uniform designs obtained}

% TODO Definir instancias e indicadores

\subsection{Testing on classical MO instances}

% TODO Definir instancias

\section{Conclusions}


\bibliographystyle{splncs04}
\bibliography{bibliografia}

\end{document}
